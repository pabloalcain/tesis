En escenarios como multifragmentación, colisiones e incluso reacciones químicas, una de las tareas más usuales es la identificación de fragmentos.
De acuerdo a la concepción de los fragmentos, estos se dividen en dos grandes grupos: los geométricos y los dinámicos.

\section{Fragmentos geométricos}
En los fragmentos geométricos lo único relevante es la disposición espacial de las partículas.
El modelo de fragmento geométrico más expandido en dinámica molecular es el llamado MST, en el que dos partículas pertenecen al mismo cluster $\{C^{\text{MST}}_n\}$ si la distancia entre ellas es menor que una distancia de corte $r_{cut}$:
\begin{equation*}
  i \in C^{\text{MST}}_n \Leftrightarrow \exists j \in C_n \mid
  r_{ij} < r_{cut}
\end{equation*}

Hay una gran cantidad de métodos para encontrar clusters geométricos, una rama de la clasificación computacional que ha crecido mucho gracias a su utilización en \emph{machine learning}.
La gran ventaja (y el motivo por el que es tan extendido el uso) de MST tiene un significado físico detrás: si la interacción entre las partículas es atractiva y las partículas tuvieran velocidad cero, todas las partículas que forman parte del cluster MST estarían ligadas energéticamente.

Sin embargo, si las partículas tienen velocidad no nula, varias partículas pueden formar parte de un mismo cluster MST por un lapso de tiempo muy breve.
Supongamos, por ejemplo, dos partículas con una interacción atractiva que están a una distancia $r<r_{cut}$ pero que tienen una velocidad relativa tal que $K_{ij} + V_{ij} > 0$.
Estas partículas forman parte del mismo cluster MST, pero luego de un tiempo $t_0$ su distancia va a aumentar hasta ser $r > r_{cut}$.
¿Es correcto decir que las partículas formaban un cluster a $t<t_0$ pero no lo forman a $t>t_0$?
Parece más adecuado decir que las dos partículas no formaban parte del mismo cluster, sino que eran simplemente dos clusters distintos que, temporariamente, estuvieron cerca.
Es por esto que muchas veces se asocian los fragmentos MST a los fragmentos \emph{instantáneos} y no fragmentos \emph{estables}.

\section{Fragmentos dinámicos}

Como dijimos, la definición de fragmento geométrico lleva a pensar en un fragmento espacial instantáneo.
Si queremos identificar los fragmentos teniendo en cuenta su dinámica, tenemos que considerar la energía cinética y potencial relativa.
El ejemplo que dimos anteriormente del sistema de dos partículas puede ser extendido a un sistema de muchas partículas.
Inspirados en los cluster MST, definimos los clusters MSTE.
En este caso, dos partículas pertenecen al mismo cluster $\{C^{\text{MSTE}}_n\}$ si están ligados energéticamente:
\begin{equation*}
  i \in C^{\text{MSTE}}_n \Leftrightarrow \exists j \in C_n : V_{ij}+ K_{ij} \le 0
\end{equation*}

A pesar de que esta definición de clusters mejora notablemente la estabilidad de los clusters con el tiempo, un simple ejemplo puede ser estudiado para ver cuán estables son los clusters MSTE.
Consideremos una simple interacción del tipo pozo de potencial, en la que
\begin{equation}
V_{ij}(r) =
\begin{cases}
-V_0 &\text{si } r \leq a\\
0 &\text{si } r > a.
\end{cases}
\end{equation}

Estudiemos ahora un conjunto de partículas de masa $m$ con posiciones $r_i = i\,a$ (con $i \in \mathcal{Z}$) tal que todas las partículas están a distancia $a$ de sus vecinos más cercanos.
Si la velocidad es $v_i = i\,v$, cada partícula va a estar unida energéticamente con sus vecinos siempre que $v \leq \sqrt{2\,V_0/m}$.
Para $2n+1$ partículas, con $-n \leq i \leq n$, la energía cinética del sistema será
\begin{align}
  K_{\text{CM}} &= \sum_{i=-n}^n \frac{1}{2} m\, i^2 v^2\\
  &= \frac{n^3}{3}\,m\,v^2 + \mathcal{O}(n^2)
\end{align}
La energía potencial es
\begin{align}
  V_{\text{CM}} &= \sum_{i=-n}^n -i V_0\\
  &= - 2\,n^2\,V_0
\end{align}
Así, para $n$ suficientemente grandes, este cluster va a ser claramente inestable en el tiempo, ya que su energía total es mayor a cero.

El modelo de clusters más estables de los desarrollados hasta ahora es el Algoritmo de Reconocimiento Temprano de Fragmentos~\cite{dorso_early_1993}, ECRA por sus siglas en inglés.
En este algoritmo, las partículas se particionan en distintos fragmentos disjuntos, $C^{\text{ECRA}}_n$, con la energía total para cada fragmento:
\begin{equation*}
  \epsilon_n = \sum_{i \in C_n} K^{CM}_i +  \sum_{i,j \in C_n} V_{ij}
\end{equation*}
donde $K^{CM}_i$ es la energía cinética relativa al centro de masa del fragmento.
El conjunto de fragmentos $\{C_n\}$ es, entonces, el que minimiza la suma de todas las energías de los fragmentos $E_{\text{partición}} = \sum_n \epsilon_n$.

El enfoque de este algoritmo es muy distinto a los anteriores ya que no es aditivo.
Esto hace necesario que las posibles distribuciones de fragmentos tengan que ser estudiadas caso por caso, y la solución entonces pertenece al universo de la optimización combinatoria.
Para encontrar soluciones aproximadas, el método propuesto originalmente es similar al Recocido Simulado~\cite{dorso_early_1993}.
Para sistemas grandes este método tarda mucho en converger, por lo que utilizamos el Modelo de Fusión Binaria desarrollado por Puente~\cite{puente_efficient_1999}.
En este modelo, partimos de una configuración con todas las partículas en un cluster distinto.
Partiendo de esta situación, en la que $E_{\text{partición}}^0 = 0$, seguimos los siguientes pasos:
\begin{enumerate}
\item Explorar todas las posibles fusiones de dos clusters y calcular la energía de partición $E_{\text{partición}}^{i+1}$ de cada una de ellas.
\item Elegir la fusión que resulte en la menor $E_{\text{partición}}^{i+1}$.
\item Si $E_{\text{partición}}^{i+1} < E_{\text{partición}}^i$, realizar la fusión y volver al paso inicial; si no, detener la iteración.
\end{enumerate}

Este método da, para sistemas grandes, resultados muy similares al resultado exacto de ECRA.\footnote{Este método es muy similar a la \emph{2-opt} para resolver aproximadamente el problema del viajante}
Justamente por la \emph{estabilidad} de estos clusters es que identifican clusters \emph{asintóticos}; para tiempos muy largos (una vez que la dinámica ya está resulta en gran parte y sólo queda la dinámica interna de cada cluster) todos estos modelos de clusters identifican los mismos estados.