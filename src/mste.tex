Un simple ejemplo puede ser estudiado para ver si los clusters MSTE son estables o no.
Consideremos una simple interacción del tipo pozo de potencial, en la que
\begin{equation}
V_{ij}(r) =
\begin{cases}
-V_0 &\text{if } r \leq a\\[2ex]
0 &\text{if } r > a.
\end{cases}
\end{equation}

Estudiemos ahora un conjunto de partículas de masa $m$ con posiciones $r_i = i\,a$ (con $i \in \mathcal{Z}$) tal que todas las partícuals están a distancia $a$ de sus vecinos más cercanos.
Si la velocidad es $v_i = i\,v$, cada parícula va a estar unida energéticamente con sus vecinos siempre que $v \leq \sqrt{2\,V_0/m}$.
Para $2n+1$ partículas, con $-n \leq i \leq n$, la energía cinética del sistema será
\begin{align}
  K_{\text{CM}} &= \sum_{i=-n}^n \frac{1}{2} m\, i^2 v^2\\
  &= \frac{n^3}{3}\,m\,v^2 + \mathcal{O}(n^2)
\end{align}

La energía potencial, sin embargo, es
\begin{align}
  V_{\text{CM}} &= \sum_{i=-n}^n -i V_0\\
  &= - 2\,n^2\,V_0
\end{align}
