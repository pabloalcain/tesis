%COULOMB

A lo largo de esta tesis, dedicamos gran parte del trabajo a la ecuación de estado de la materia de estrellas de neutrones en condiciones de equilibrio a través de simulaciones computacionales de dinámica molecular, con detalle al nivel de los protones y neutrones.
El primer paso para describirla fue estudiar el rol del término electrostático de Coulomb en ella.
Uno de los modelos más utilizados para el término de Coulomb es el de la aproximación de Thomas-Fermi, en el que la interacción de Coulomb está apantallada con una longitud de apantallamiento $\lambda_D$ que, de acuerdo a cálculos teóricos, es $\lambda\approx100\,\text{fm}$.
Sin embargo, en simulaciones con detalle a nivel de los nucleones (como las que utilizamos en este trabajo), este valor es prohibitivo y fue histórica y arbitrariamente reducido a $\lambda\approx10\,\text{fm}$, con la expectativa de mantener la fenomenología básica cuando se estudiaban sistemas pequeños.
Encontramos, sin embargo, que existe una longitud de apantallamiento crítica $\lambda_c$ a partir de la cual la estructura del estado fundamental cambia drásticamente.
Para $\lambda<\lambda_c$, las estructuras no homogéneas que emergen de las simulaciones se deben a efectos de tamaño finito, efecto que se nota a partir de la presión negativa de estas estructuras y la presencia de una estructura por celda.
Para $\lambda>\lambda_c$ la presión se vuelve positiva y los sistemas presentan fluctuaciones de densidad a una escala menor que la de la celda, pero no con la forma de la pasta típica.
Este régimen de transición se caracteriza por grandes fluctuaciones en la superficie, ancho medio y característica de Euler $\chi$ de las estructuras.
Es recién para $\lambda=20\,\text{fm}$ que la morfología de las estructuras se estabiliza y deja de depender de $\lambda$ y reproducen las pastas usuales.
A partir de este resultado, utilizamos $\lambda=20\,\text{fm}$ como valor para la longitud de apantallamiento, ya que es factible computacionalmente y reproduce la física de la aproximación de Thomas-Fermi.

%TRANSICIONES DE FASE

Estudiamos también la dependencia de la materia de estrellas de neutrones para temperaturas distintas de cero, y encontramos una transición de fase de tipo sólido-líquido para todas las densidades estudiadas a temperaturas muy bajas, caracterizada a través del coeficiente de Lindemann y las curvas calóricas.
Además, esta transición está marcada por una discontinuidad en los funcionales de Minkowski.
La transición no altera la forma típica de la pasta: la fase líquida preserva la forma de la pasta encontrada en la fase sólida.

Para temperaturas superiores a $T\approx0.7\,\text{MeV}$, las formas típicas de la pasta se vuelven inestables y el sistema adopta estructuras más desordenadas, pero aún heterogéneas.
Sin embargo, el pico del factor de estructura de bajo momento en estas estructuras se mantiene bastante elevado: las formas tradicionales de pasta no son necesarias para tener un alto factor de estructura y, en consecuencia, un alto scattering de neutrinos energéticos.
De este modo, la existencia de formas tradicionales de pasta --que sólo se obtienen a temperaturas muy bajas-- no es una condición necesaria para el scattering de neutrinos en la corteza de las estrellas de neutrones.

Más aún, encontramos que a $T\sim0.7,\text{MeV}$ el sistema puede existir en varios estados estables, todos con distinta morfología, pero con energías internas muy cercanas.
A partir de nuestras simulaciones a $(N,V,T)$ fijos, estos estados parecen estar separados por barreras de energía relativamente altas, que hacen que la transición entre ellos sea un evento poco probable, no observado a partir de una sola simulación.
De todos modos, las distribuciones de energía obtenidas a partir de distintas condiciones iniciales se solapan considerablemente, indicando que cualquiera de ellos puede ser el estado de equilibrio a esa temperatura.
Todo esto indica que el estado real de estos sistemas a temperaturas bajas, pero finitas, debe ser descripto a través del estudio de un ensamble de estructuras en vez de una sola estructura tipo pasta.

%NUEVAS PASTAS

Para realizar un estudio más detallado de estas pastas no tradicionales y de su scattering, estudiamos el comportamiento del factor de estructura de neutrón-neutrón y la función de distribución de pares para un gran rango de densidades, temperaturas y fracción de protones.
Observamos que para densidades altas $\rho>0.03\,\text{fm}^{-3}$, donde esperamos que aparezcan fragmentos muy grandes (\emph{spaghetti} u \emph{lasagna}), la longitud de onda del pico de opacidad se mantiene relativamente constante, y la máxima opacidad se obtiene para neutrinos muy energéticos ($E_\nu \approx 80\,\text{MeV}$, típicos de una muy inicial etapa de evolución de las proto-estrellas de neutrones).
A medida que disminuye la densidad, nos movemos hacia la fase de \emph{gnocchi}, en la cual los fragmentos son de tamaño finito y la máxima opacidad se mueve a energías menores.
Este aumento en la opacidad no sólo se produce cuando las heterogeneidades forman parte de las comúnmente conocidas como pasta nuclear, sino también cuando son bastante diferentes (la \emph{pasta nuclear generalizada}).
También es relevante recordar que la opacidad no se produce para una longitud de onda en particular, sino que para un espectro bastante amplio de ellas.

Es de esperar que estos resultados sean cualitativamente correctos, pero que dependan cuantitativamente del modelo escogido para describir la materia rica en neutrones.
El modelo que estamos utilizando en este trabajo fue puesto a prueba extensivamente en colisiones y en física de iones pesados; es por esta razón que lo escogimos para describir cuantitativamente la materia rica de neutrones.

% FRAGMENTACION

Finalmente, pasamos a estudiar el comportamiento de la materia de estrellas de neutrones fuera del equilibrio.
Específicamente, la formación de fragmentos a partir de experimentos numéricos de sistemas expandidos homogéneamente.
Para analizar la estructura del sistema en función del tiempo, desarrollamos una herramienta basada en análisis de grafos para identificar fragmentos infinitos para cualquier tipo de fragmentos aditivos.
Una vez que esta formalismo fue aplicado a las simulaciones mencionadas, pudimos identificar la región en la que se da una distribución de fragmentos de tipo ley de potencias.
Esta distribución tiene regímenes desde U-shaped hasta decaimiento exponencial.

Para poder modelar la fragmentación para diferentes parametrizaciones del potencial, estudiamos las propiedades estructurales de la corteza de estrellas de neutrones a través de tres potenciales distintos.
Estos potenciales involucran un término nuclear escogido para reproducir energías de ligadura y compresibilidad de la materia nuclear, sumado a la interacción de Coulomb.
Para analizar las estructuras que se forman utilizamos cuatro tipos distintos de algoritmos reconocedores de fragmentos: MST, MSTE, MSTpC y ECRA-BFM.\@
Con estos algoritmos encontramos que de los tres potenciales, dos de ellos (New Medium y SSP) desarrollaron una estructura nueva para bajas fracciones de protones que llamamos \emph{pregnocchi}.
Esta estructura consiste en agregados de protones formados por la mediación del término atractivo $V_{np}$ del potencial que resistieron la expansión.

También analizamos la expansión de materia rica en neutrones infinita descripta a través del modelo del pequeño \emph{big bang}.
Mostramos que, en general, la identificación adecuada de la estructura depende mucho del algoritmo escogido, siendo ECRA y MSTpC los más adecuados para encontrar las estructuras y ECRA el más estable.
Este enfoque, combinado con distintos algoritmos reconocedores de fragmentos, nos permitió identificar la dinámica de la formación de fragmentos.
Un análisis cuidadoso de la dinámica de formación de fragmentos mostró que se formaron muy temprano en la expansión.
En particular, la novedosa estructura que llamamos \emph{pregnocchi} es de considerable relevancia, ya que de acuerdo al análisis de ECRA estos agregados preexistentes evolucionan en una nube de neutrones, resultando en configuraciones en las que el \emph{r-process} se puede desarrollar.


\subsection{Perspectivas de futuro}

A lo largo de la producción de esta tesis surgieron distintos interrogantes que forman parte de las perspectivas de trabajo futuro.

\subsubsection{Parametrización de la interacción nuclear: Pauli}

Como ya vimos en los resultados mostrados, muchos de los resultados dependen (con mayor o menor sensibilidad) del modelo elegido para la parametrización.
Quizás el caso más extremo que vimos es la presencia de \emph{pre-gnocchi} para una parametrización específica, mientras que no lo encontramos para otras.
Hay otras diferencias también más sutiles: entre ellas el valor de $\lambda_c$ del capítulo~\ref{ch:coulomb} o las temperaturas de transición sólido-líquido que vimos en el capítulo~\ref{ch:transicion}


\subsubsection{Opacidad a neutrinos de estrellas}

Los modelos hidrodinámicos para la materia rica en neutrones~\cite{ruffert_coalescing_1995, mezzacappa_investigation_1998, geppert_temperature_2004, woosley_physics_2005, liebendorfer_supernova_2005} pueden sugerir fracciones de protones, densidades y temperaturas para distintas condiciones (supernovas, proto-estrellas de neutrones, estrellas de neutrones).
A partir de trabajos microscópicos podemos hallar la opacidad, distribución de neutrones (y consecuentemente probabilidad de \emph{r-process}) en función de $(\rho, T, x)$ e insertar las predicciones de cada modelo en los cálculos hidrodinámicos.
Combinando estos dos resultados con mediciones eventuales de la opacidad de los neutrinos en estrellas de neutrones, podemos comprobar la validez de distintos modelos nucleares y, en consecuencia, avanzar hacia la ecuación de estado de la materia nuclear.

\subsubsection{¿Existen las pastas tradicionales?}
